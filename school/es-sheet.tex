\RequirePackage{plautopatch}
\RequirePackage[l2tabu, orthodox]{nag}

\documentclass[platex,dvipdfmx]{jlreq}			% for platex
% \documentclass[uplatex,dvipdfmx]{jlreq}		% for uplatex
\usepackage{graphicx}
\usepackage{bxtexlogo}

\title{ESシート}

\author{学生番号20C1119 森田大雅}
\date{\today}
\begin{document}
\maketitle
\section{得意分野・科目・学業に力を入れていることなど、学業における強みをアピールしてください(200文字以内)}
\parindent=0pt
$\ast$ 研究テーマ(卒・修論内容)や専門的に学んだ分野でも結構です
\\
$\ast$ 研究テーマについてまだ決まっていないが書きたい場合、予定・やりたいテーマで結構です
\\
\\
私は学部生活で扱えるプログラミング言語を増やすことと、Linuxの仕組みについての勉強を主にしてきました。
数値解析の勉強でC言語の理解をより深め、Androidアプリを開発できるようにJavaを勉強し、ROSや機械学習のためにPythonを勉強しました。
またCLIの理解を深めるためLinuxを学んでいたときにシェルスクリプトについても勉強しました。(175)

\section{就職活動の軸とその理由を教えてください(200文字以内)}
\parindent=0pt
$\ast$ YKKへの志望理由は記載しないでください
\\
軸は会社の成長です。
それに必要な要素が二つあると思っています。
一つは生み出す製品にプライドを持っているかです。
これは会社の高い技術力や信頼に関わる重要なことだと考えています。
二つ目は風通しの良い環境であることです。
入社した社員はその会社に何かしらで必要だと思われたから選ばれたのであり、その力が発揮できない環境というのは、今後の会社の製品の質や結果に大きく影響すると考えるからです。(191)

\section{YKKを志望した理由を教えて(100文字以内)}
ファスナー業界の世界のトップを走り続ける技術力と信頼の高さに心惹かれました。
中国のSBSなど他を寄せ付けない圧倒的なシェアを誇るグローバル企業にこれからもしていきたいとの強い思いで志望しました。(97)

\section{YKKの中で興味のある職種、順位、理由(50文字以内)を教えてください}
\parindent=0pt
(1) 社内システムエンジニア(情報システム)
\\
学部生時に興味を持ち、勉強してきたことが情報系の分野で、活かせるものがあると思ったからです。
(46)
\\\\
(2) 電装・システム開発 
\\
画像処理を卒業研究にしようと考えており、継続して同じ分野に携われるかもしれないと思ったからです。(48)
\\\\
(3) FAロボティクス
\\
学科で様々なロボティクスに役立つ授業を受けてきたので、全体的に専攻分野に近いと思ったからです。(47)

\section{これまでで一番意欲的に取り組んだこと、またその中で一番印象に残っていることを教えてください(300文字以内)}
小中高とサッカーをやっていた。
高校は都内有数の強豪で私の代は外れと言われていたが、中でも私は誰より体力、技術、身体能力が劣っていた。
入部してすぐの頃、デイズというサッカー漫画の主人公の影響で少しずつサッカーへの姿勢が変わり、部の中で一番練習する人間になっていた。
ドベだった私が試合で一番活躍する場面が少しずつ増えるとチームのある変化に気づいた。
ドベに負けじと奮い立つメンバーが増えていたのだ。
出場はできなかったが3年時に関東大会やインハイ、選手権全国と歴代で屈指の成績を残す代となった。
このとき私は落ちこぼれが頑張る姿に周囲は影響され全体が良くなっていくのを直接経験でき、印象に残る高校生活となった。(300)


\section{周囲と協働して取り組んだ経験を教えてください(300文字以内)}
研究室配属を控えた新3年生に向けて学科の研究室紹介を行う行事だ。
準備期間が一週間と短い中でポスター制作、動作確認、発表練習をする。
私は全体会議や当日の設営、撤収をする研究室の代表を務めた。
私も含めホストの新4年は就活やバイトで忙しいが、十人近くいる研究室メンバーの中には連絡をスルーして準備に来ない人が結構いて、一人分の負担が増え大変だった。
私が進捗管理もし、集まれる4年三人で隙間時間も活用しながら準備した。
当日は発熱で設営のあと途中退場になった。
展示として体を成し役目を全うできたかもしれないが、繰り返し連絡をいれたメンバーの協力は最後まで得られず、今回はその点代表としての難しさを経験した。(298)

\end{document}
